\documentclass{article}
\usepackage{graphicx}

\begin{document}

\title{A Study about Cloud Computing Services in Smart Learning System}
\author{Guntur Dharma Putra}
\date{}

\maketitle

\begin{abstract}
Cloud computing service offers some advantages in its implementation to e-learning system, for example, increased cost savings and also improved efficiency and convenience of educational services. Furthermore, e-learning services can be also enhanced to be smarter and more efficient using context-aware technologies since context-aware services is based on the user’s behavior. To implement  the technologies into current e-learning services, a service architecture model is needed to transform the existing e-learning environment that is only situation-aware, into the environment that also understands context. The rationale behind this paper is to study the existence or lack of existing approaches regarding the implementation of cloud computing services in smart learning system. This is done by surveying the state of the art in the area, and illustrating the requirements of context-aware smart learning system with regard to some important factors: context-awareness, security, ontology, multi-device support, and flexibility. This paper is eager to help investigating the works that have been done before for cloud computing services in smart learning system and to show the possible requirements for the future smart learning system.
\end{abstract}
\smallskip
\noindent \textbf{Keywords --} e-learning, smart learning services, cloud computing, context-aware, Internet enabled learning.

\end{document}