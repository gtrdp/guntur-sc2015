\documentclass[journal]{vgtc}                % final (journal style)
% \documentclass[review,journal]{vgtc}         % review (journal style)
%\documentclass[widereview]{vgtc}             % wide-spaced review
%\documentclass[preprint,journal]{vgtc}       % preprint (journal style)
%\documentclass[electronic,journal]{vgtc}     % electronic version, journal
\let\ifpdf\relax

%% Uncomment one of the lines above depending on where your paper is
%% in the conference process. ``review'' and ``widereview'' are for review
%% submission, ``preprint'' is for pre-publication, and the final version
%% doesn't use a specific qualifier. Further, ``electronic'' includes
%% hyperreferences for more convenient online viewing.

%% Please use one of the ``review'' options in combination with the
%% assigned online id (see below) ONLY if your paper uses a double blind
%% review process. Some conferences, like IEEE Vis and InfoVis, have NOT
%% in the past.

%% Please note that the use of figures is not permitted on the first page
%% of the journal version.  Figures should begin on the second page and be
%% in CMYK or Grey scale format, otherwise, colour shifting may occur
%% during the printing process.  Papers submitted with figures on the
%% first page will be refused.

%% These three lines bring in essential packages: ``mathptmx'' for Type 1
%% typefaces, ``graphicx'' for inclusion of EPS figures. and ``times''
%% for proper handling of the times font family.

\usepackage{mathptmx}
\usepackage{graphicx}
\usepackage{times}

%% We encourage the use of mathptmx for consistent usage of times font
%% throughout the proceedings. However, if you encounter conflicts
%% with other math-related packages, you may want to disable it.

%% If you are submitting a paper to a conference for review with a double
%% blind reviewing process, please replace the value ``0'' below with your
%% OnlineID. Otherwise, you may safely leave it at ``0''.
\onlineid{0}

%% declare the category of your paper, only shown in review mode
\vgtccategory{Research}

%% allow for this line if you want the electronic option to work properly
\vgtcinsertpkg

%% In preprint mode you may define your own headline.
%\preprinttext{To appear in an IEEE VGTC sponsored conference.}

%% Paper title.

\title{A Study about Cloud Computing Services in Smart Learning System}

%% This is how authors are specified in the journal style

%% indicate IEEE Member or Student Member in form indicated below
\author{Guntur Dharma Putra}
\authorfooter{
%% insert punctuation at end of each item
\item
  Guntur Dharma Putra is a Master Student Computing Science at the RuG, e-mail: g.d.putra@student.rug.nl.
}

%% A teaser figure is NOT to be included.

%other entries to be set up for journal
\shortauthortitle{Biv \MakeLowercase{\textit{et al.}}: A Study about Cloud Computing Services in Smart Learning System}
%\shortauthortitle{Firstauthor \MakeLowercase{\textit{et al.}}: Paper Title}

%% Abstract section.
\abstract{Cloud computing services offer several benefits in its implementation to e-learning system, such as increased cost savings and also improved efficiency and convenience of educational services. Furthermore, e-learning services can be also enhanced to be smarter and more efficient using context-aware technologies as context-aware services are based on the user’s behavior. To add those technologies into existing e-learning services, a service architecture model is needed to transform the existing e-learning environment, which is situation-aware, into the environment that understands context as well. The rationale behind this paper is to study the existence or lack of existing approaches regarding the implementation of cloud computing services in smart learning system. This is done by surveying the state of the art in the area, and illustrating the requirements of context-aware smart learning system with regard to some important factors: dynamicity, scalability, dependability, security and privacy. This paper is eager to help investigating the works that have been done before for cloud computing services in smart learning system and to show the possible requirements for the future smart learning system.
} % end of abstract

%% Keywords that describe your work. Will show as 'Index Terms' in journal
%% please capitalize first letter and insert punctuation after last keyword
\keywords{e-learning, smart learning services, cloud computing, context-aware, Internet enabled learning.}

%% ACM Computing Classification System (CCS). 
%% See <http://www.acm.org/class/1998/> for details.
%% The ``\CCScat'' command takes four arguments.

\CCScatlist{ % not used in journal version
  \CCScat{K.6.1}{Management of Computing and Information Systems}%
{Project and People Management}{Life Cycle};
  \CCScat{K.7.m}{The Computing Profession}{Miscellaneous}{Ethics}
}

%% Copyright space is enabled by default as required by guidelines.
%% It is disabled by the 'review' option or via the following command:
% \nocopyrightspace

%%%%%%%%%%%%%%%%%%%%%%%%%%%%%%%%%%%%%%%%%%%%%%%%%%%%%%%%%%%%%%%%
%%%%%%%%%%%%%%%%%%%%%% START OF THE PAPER %%%%%%%%%%%%%%%%%%%%%%
%%%%%%%%%%%%%%%%%%%%%%%%%%%%%%%%%%%%%%%%%%%%%%%%%%%%%%%%%%%%%%%%%

\begin{document}

%% The ``\maketitle'' command must be the first command after the
%% ``\begin{document}'' command. It prepares and prints the title block.

%% the only exception to this rule is the \firstsection command
\firstsection{Introduction}
\maketitle

By the turn of the century, the fast development of digital technologies is creating not only new opportunities for our society but challenges to it as well. Our society is now being reshaped by rapid advances by technologies in the field of education, telecommunications, sciences and many more. Today, e-learning and cloud computing is emerging as the new-fangled paradigm of modern education with reduced upfront investment for teachers and the apprentices. E-learning is an Internet based learning process, using Internet technology to design, implement, select, manage, support and extend learning, which will not replace traditional educational methods, but will greatly improve the efficiency of higher education \cite{SudhirKumarSharmaNidhiGoyal2014}.

An increasing number of universities and educational institutions in the USA and UK are adopting cloud computing not only for increased cost savings but also for improving the efficiency and convenience of educational services. The cloud computing systems have been conducted for e-learning services. However, most of the current cloud-based education systems focus on delivering and sharing learning materials rather than supporting and establishing an integrated, total cloud-based educational service environment.

% explain a bit about facebook in education
Both time spent on Facebook and time spent engaged in certain Facebook activities can be positively predictive, negatively predictive, or positively and negatively predictive of engagement, depending on the outcome variable. For example, time spent on Facebook is positively predictive of time spent in co-curricular activities while playing games on Facebook is negatively predictive. These results are congruent with others that have found that using the Internet and Facebook (Ellison et al., 2011) in certainways leads to better psychosocial outcomes, and that using Twitter (Junco, Heiberger et al., 2010) in certainways leads to better academic outcomes. Therefore, Facebook use in and of itself is not detrimental to academic outcomes, and can indeed be used inways that are advantageous to students.

Higher education administrators, faculty and staff have an opportunity to help students use Facebook inways that are beneficial to their engagement and, by extension, to their overall academic experience. Given that Facebook continues to be popular among college students, and that universities are interested in engaging and retaining students, it is important for those working in higher education to familiarize themselves with Facebook (and other such technologies) and to design and support interventions that meet students where they aredin order to help them get to where they are going \cite{Junco2012}.

% draw an introduction about smart computing in general

The rest of this review paper is organized as follows. Section 2 starts with general introduction into middleware. Section 3 illustrates the recent protocols and standards applied at middleware. Section 4 shows the categorization of middleware technologies according to the context of the study. Section 5 provides illustration for the main requirements of smart spaces with regard to middleware technology for services in ubiquitous computing. Section 6 gives some practices of current smart homes projects. Section 7 studies the middleware of SM4ALL and the rest of the practices described in section 6, according to the characteristics given in section 5. Section 8 describes the future directions of research work within the middleware for smart homes. Section 9 is the research conclusion.

%% \section{Introduction} %for journal use above \firstsection{..} instead

\section{E-Learning and Smart Learning}
% conventional e-learning
Although e-learning is being used more intensively in recent years and some professionals have demonstrated a willingness to explore new approaches, many organizations still hold reservations about becoming involved with innovative pedagogical tools and have not yet realized what can be achieved with them. The range of possibilities offered by e-learning has not been fully exploited. For example, some organizations continue to limit their use of such tools to their repository functions, perhaps pushed to do so because of the poor IT skills of their employees. 

Despite its limitations, this study has contributed to research and practice in e-learning adoption. The results revealed the impact of factors such as performance expectancy and individual-level social influence on the continuance intention of e-learning and its effects on their performance. Digital literacy as a construct deserves more attention in e-learning and other settings because it incorporates the idea of IT use as a skill that evolves. In light of these findings, the study has offered various suggestions to different communities of practitioners to improve their performance with regards to the adoption and continued use of e-learning \cite{Mohammadyari2014}.

% explain about e learning online learning distance learning
The definitions found in various articles mirror the conflicting responses provided by the respondents in this study. The lack of consistency in terminology inevitably affects not only the researchers who would like to build upon the findings, but also impacts designers who are creating similar types of environments. Terminology also poses a problem when the specific context of the learning environ- ment is not described in sufficient detail or when its identification is not very prominent in both the discussion of the methods and the other sections of the paper. This not only impacts the evaluation of such learning experiences but also the future of successfully delivered distance learning events. The findings show great differences in the meaning of foundational terms that are used in the field, but also provide implications internationally for the referencing, sharing, and the collaboration of results detailed in varying research studies \cite{Moore2011}.

% Smart Learning
The smart learning (s-learning) has become an important way of learning during the last few years \cite{Kim2013}. It has been made possible by the recent advancements in the Mobile Internet and Information technologies. The S-learning has a major role in creating a good and personalized learning environment, and also being well adapted to the current education model wherever possible \cite{Uden2007}. Usually, the teaching and learning that e-learning offers is only inside of a lecture-style classroom with desktop computers. Though the students could download information and browse through the existing e-learning platform regardless of time and place, they were still confined to the limits of the wired classroom-setups.

The concept of s-learning plays an important role in the creation of an efficient learning environment that offers personalized contents and easy adaptation to current education model. It also provides learners with a convenient communication environment and rich resources. However, the existing-learning infrastructure is still not complete. For example, it does not allocate necessary computing resources for s-learning system dynamically \cite{Kim2013}. Currently, the majority of s-learning systems have difficulty in interfacing and sharing data with other systems, i.e., it falls short of systematic arrangement, digestion and absorption of the learning contents in other systems. This may lead to duplication in creating teaching resources and low utilization of existing resources. To resolve this problem, it is recommended to use cloud computing to support resource management.

So far there is no clear definition of Smart Learning. Related scholars and people who are involved with education business are discussing that the concept of Smart Learning should not be limited to just utilizing smart devices. Thus, the government, academia, and the educational industry have been working on defining and categorizing Smart Learning. At the Smart Learning Korea forum 2010 [9], a concept of Smart Learning was proposed as follows: first, it is focused on humans and content more than on devices; second, it is effective, intelligent tailored-learning based on advanced IT infrastructure [10]. Also, Kwak Duk-hoon, president and CEO of EBS (Korea’s Educational Broadcasting System), said the term \"Smart Learning\" was first used in Korea. Then, what is the broad concept of Smart Learning in Korea? \cite{Kim2013a}.

\section{Cloud Computing and Education}
This section describes about the relation between cloud computing and education subject. The education mentioned in this section means e-learning and s-learning as well.

% explain a little bit about mooc
In recent years several studies applied the first principles of instruction to evaluate online and blended learning courses in a variety of contexts and domains. In the future, researchers could use these principles (and the Course Scan) to carry out systematic compara- tive studies of instructional quality of different types of courses. Furthermore, the ten-principle framework could be applied to analyse a larger sample of MOOCs from a broader range of platforms and countries. Future analysis could detail distinctive characteristics of MOOCs, which scored highly on some of the principles, identifying similarities and patterns in these characteristics. xMOOCs delivered through different commercial platforms could be compared to ascertain any significant differences and patterns in their designs. Finally, future studies could examine interrelationships between instructional design quality (as determined by Course Scan results) and other aspects of course quality, such as improved learning and learner satisfaction (as determined by other types of course evaluation) \cite{Margaryan2014}.

% still about mooc
An important finding of this study is that perceived reputation and perceived openness are the two strongest predictors to explain MOOCs continuance intention to use. This finding is unique in the context of MOOCs research. Confirmation of user expectations had the greatest influence on user satisfaction. Additional research which explores the motivations of individuals that participate in and continue using MOOCs is warranted and has value for both researchers and practitioners. The research findings affirm for MOOC providers the importance of openness and reputation for future MOOCs course offerings. Openness and reputation are ways that MOOC providers can both differentiate themselves from competitors and enhance an individual's intention for continued MOOCs enrollment \cite{Alraimi2014}.

% exlpain a bit about business paradigm in e learning
Cloud Computing: A New Business Paradigm for E-learning \cite{Laisheng2011}.

  \subsection{Necessity of Cloud Computing in Educational System}
  Cloud computing is reducing the difference between on campus education and distance education still there are few limitations of E-learning for Lab based education due to computation power. Fortunately cloud computing is the technology which can offer different services in three layers. cloud computing enable students to access the knowledge by sharing distributed E-learning resources in a public, private or hybrid cloud systems. Due to using cloud computing system for deploying a modern educational systems, universities and other organizations must take to account various items such as cost and accelerate delivery of learning service, quickly learning, and privacy issue. Therefore, cloud service providers should especially attended to offering cloud-based learning for improving education status in poor countries in Asia and Africa \cite{Bouyer2014}.

  \subsection{Cloud Computing and Smart Learning}
  The cloud computing is defined as a technology that provides its users with IT resources by using the Internet as a medium. The users can use IT resources such as application software or storage space from the cloud without needing to own them. The users only need to pay per usage charges for the resources they used. The concept of cloud computing is not new. It is the combination of distributed computing, grid computing, utility computing, etc. [3-5]. When a user requests services from some cloud server, the server immediately provides the requested services to the user based on their request details. It means that the cloud computing has the ability to customize its service to each user. Since servers charge fees based on usage, it can automatically guide its users of their service request based on previous usages. These features allow that users to use the service only the amount they need at their desired time. Also, there are numerous cloud based applications that are freely available and the trend for that continues to grow [6,7]. There are a number of cloud-based applications available in the e-learning sector as well \cite{s110807835}.
  
  Casquero et al. [8] presented a framework based on iGoogle and using the Google Apps infrastructure for the development of a network of cooperative personal learning environments. They discussed the integration of institutional and external services in order to provide customized support to faculty members in their daily activities. They also take advantage of the framework as a test-bed for the research, implementation and testing of their educational purpose services. Marenzi et al. [9] investigated how educational software can be used in an academic or corporate learning environment. They integrated models and tools that they developed into an open source environment for the creation, storage and exchange of learning objects as well as learning experiences. They presented the “LearnWeb 2.0” infrastructure to support lifelong learning and to enhance the learning experience. This infrastructure brings together information stored on institutional servers, centralized repositories, learners’ desktops, and online community—sharing systems like Flickr and YouTube. Sedayao [10] proposed an online virtual computing lab that offers virtual computers equipped with numerous applications such as Matlab, Maple, SAS, and many others that can be remotely accessed from the Internet.
  %insert some tables here

\section{Cloud Computing System in Smart Learning Approaches}
  \subsection{Smart Cloud Computing}
  The Smart Cloud Computing (SCC for short) based on elastic computing for 4S model has the capability to provide a smart learning environment. It encourages learning system standardization and provides a means for managing it. A traditional e-learning system can display single content on a single device or multiple contents on one device. The SCC can deliver s-learning to the users so they can use multiple devices to render multi learning contents. The multi learning contents can be played in different devices separately to form a “virtual class”. For this, the SCC uses context-aware sensing. Sensing through the location and IP address of each device, it can orchestrate all devices. The architecture of the model is shown in the Figure 2 \cite{s110807835}.

  \subsection{Elastic Model}
  % This is WRONG, please change this
  The smart learning (s-learning) has become an important way of learning during the last few years \cite{Kim2013}. It has been made possible by the recent advancements in the Mobile Internet and Information technologies. The S-learning has a major role in creating a good and personalized learning environment, and also being well adapted to the current education model wherever possible \cite{Uden2007}. Usually, the teaching and learning that e-learning offers is only inside of a lecture-style classroom with desktop computers. Though the students could download information and browse through the existing e-learning platform regardless of time and place, they were still confined to the limits of the wired classroom-setups.
  
  \subsection{Context-aware}
  Context-aware is important \cite{Pratama2014}.

  Despite advances in mobile, interactive, and ubiquitous technology, the adoption of instructional technology and learning aids in traditional learning spaces has been lackluster. The system described in this paper is designed from the ground up to be affordable to implement, easy to operate, and most importantly, it is designed to keep the focus on the learning material by offering flexible and efficient learning interactions. It includes a cost-effective room-level location detection system based on IR light. This provides context awareness and allows the system to dynamically adapt based on its surroundings. Finally, the system provides efficient knowledge dissemination and sharing, and collaboration capabilities based on Semantic Web technology. It is designed to enhance traditional learning situations with modern instructional technology including handheld devices. The system has the potential to expand beyond indoor learning spaces through the use of mobile technology and other communication systems, thus providing a completely comprehensive learning experience \cite{Scott2010}.
  
  \subsection{Ontology}
  The work of \cite{nasr2012proposed} has described this approach.

  There have been many new advances in the computing field in recent times. Cloud Computing and Web 3.0 are two such areas that are beginning to significantly impact how we develop, deploy and use e-Learning application. Web 3.0 combines semantic Web with Web 2.0’s tagging culture. It will use internet to make connections with in- formation. Cloud Computing presents a new way of de- ploying applications. Today we can get Infrastructure as a Service (IaaS), Platform as a Service (PaaS) or Software as a Service (SaaS). There are elastic clouds where mem- ory and processing power get allocated based on comput- ing resources required at the time. Moreover, learning environment must be productive, scalable, flexible and adaptable towards learners’ needs and learner preferred information and communication technologies. This raises the question of whether cloud computing and web3.0 can meet the indicated requirements. To answer this question, Figure 2 introduces the proposed model with new tech- nologies, integration between cloud computing and web 3.0.

  % MONTO: ontology based learning -> explain about this
  In summary, the MONTO ontology fits very well into the general architecture of the Intelligent Tutoring Systems (pedagogy ontology to pedagogy module, Domain Ontology to expert module, task ontology to problem-solving environment, and student model ontology to student module) that can be considered as part of a smart learning environment. The MONTO ontology at the back-end is used for eight critical aspects (mentioned in ‘‘The functionalities derived from ontology’’ section) of the smart learning environment. It is concluded that the use of the MONTO-based system bridges the gaps highlighted in the state of the art study (See Fig. 14). A more effective implementation (constraint-free text entry, different way of presentation of schematic and semantic knowledge, inclusion of diagrammatic tool for drawing diagrams) would make the system more user-friendly. This ontology can also be used for drawing diagrams automatically for a given problem by using the problem schema. Those diagrams can be matched with the diagrams drawn by students to find missing concepts and misconceptions of the students. Using natural language processing (NLP) and machine learning (ML) techniques, one can generate problems similar to the problems being modeled in the ontology and create analytical questions for a given word problem that can be used in the pedagogy ontology. A completely functional system could use ML techniques to learn how problems are solved by good students and the same strategies could be taught to weak students. MONTO can be used as a part of a bigger adaptive smart learning environment to teach mathematics. With student profiles captured in MONTO, many more learning analytics can be generated for individual students and groups of students \cite{Lalingkar2014}.

  \subsection{Content Oriented}
  This paper proposed a content-oriented smart education system based on cloud computing that integrates a number of features required for implementing a cloud-based educational media service environment. The aim was to develop a total, integrated education content service system based on cloud computing to deliver and share a variety of enhanced forms of educational content, such as text, images, video, audio, animations, memos, quizzes, and 3D and AR objects. For the realization of a cloud- based content service system, we developed six main features. First, by leveraging several IT and cloud computing technologies, we established a private cloud platform to install and operate a cloud-based educational media service environment. In addition, we identified the software and applications required for the proposed cloud-based education services. Second, we developed a common file format enabling manipulation of various forms of media content on multiple platforms using XML, WebGL, and HTML5. Third, we implemented an authoring tool, allowing teachers to create various types of smart media content, including text, images, pictures, videos, and 3D and AR objects. Fourth, we developed a content viewer to display media content on diverse types of devices, such as PCs, notebooks, netbooks, tablets, smart TVs, and smartphones, through a multi-platform based design. Fifth, we implemented an inference engine to provide students with customized individual learning content by analyzing their learning and content usage patterns. Sixth, a security system was included in the proposed system to encrypt data and to control user access for dependable smart media content services \cite{jeong2013content} \cite{jeong2013cloud}.

% disguise this part, bcs this is really important part, dont mess with it
\section{Discussion}
% describe: what drives a successful e-learning comprehensively
% and reflects to the papers.
This section draws the advantages and drawbacks of the reviewed works. This section will also reflect to the factors that drive a successful e-learning \cite{Sun2008}.

  \subsection{Advantages of Cloud Computing in Education}
  Some benefits are drawn based on \cite{Gonzalez-Martinez2014}.

    \subsubsection{A wealth of online application to support education}
    \subsubsection{Flexible creation of learning environments}
    \subsubsection{Support for mobile learning}
    \subsubsection{Computing-intensive support for teaching, learning, and evaluation}
    \subsubsection{Scalability of learning systems and applications}
    \subsubsection{Costs saving in hardware}
    \subsubsection{Cost saving in software}
  
  \subsection{Drawbacks}

\section{Concluding Remarks}
Smart Learning can be improved using smart cloud computing.

Since the main goal for this paper is to study whether the existing middleware practices at smart homes support the requirements of ubiquitous computing environments or not. This paper concludes that there are good smart homes infrastructures and practices e.g., SM4All and Gaia projects, that consider most of the basics of the ubiquitous computing needs (e.g., dynamicity, scalability, depend- ability, security and/or privacy). This done by first introducing the concept of middleware and its protocols and standards which are relevant for the projects e.g., UPnP and OSGi. Then, I classify middleware according to three types: object-oriented, service based, and object oriented base. Following this, a definition of requirements for smart spaces with regard to middleware technology are highlighted. These requirements are kept generic (e.g., dynamicity, dependability, scalability, security, and privacy) in order to allow the identification of criteria for the evaluation of middleware technology.

After that, I choose SM4ALL, RUNES, ANGEL, GTSH, Gaia, and Serenity as examples of the current practice middleware technologies projects at smart homes, discussing them with their comparisons for the middleware part according to the mentioned classification and the given ubiquitous requirements. The findings are that some projects touched the topics of dynamicity, scalability, and dependability at the middleware-service level. Also, a few of them filled the security and privacy needs. The only project that covers all defined requirements is SM4All.

However, the horizons of the present practices middleware will have to expand more and more to cater for the new requirements of ubiquitous computing (e.g., support of heterogeneity, mobility, tolerance for component failures, controlling and traceability, and ease of deployment and configuration) if they want to survive in the emerging smart space environments. This study help giving a look of the existing works of middleware at smart homes, and showing the requirements for the future vision. Moreover, it is not clear at this point exactly what smart space middleware will consist of. What is likely, however, is that various smart space architectures will emerge independently of each other which greatly increases the need for middleware to provide interoperability between heterogeneous systems.

This paper has compared and reviewed several researches and articles about cloud computing services in smart learning environment.

% \acknowledgements{
% The authors wish to thank Dr. for the help on this work.}

\bibliographystyle{unsrt}
%%use following if all content of bibtex file should be shown
% \nocite{*}
\bibliography{sc2015}
\end{document}
